\documentclass[11pt]{amsart}

\usepackage{geometry}
\geometry{letterpaper,left=1in,right=1in,top=1in,bottom=1in, includehead}
%\setlength{\textheight}{9in}


\usepackage[parfill]{parskip}
\usepackage{graphicx}
\usepackage{float}
\usepackage{amsmath}
\usepackage{amssymb}
\usepackage{amsthm}
\usepackage{array}
\usepackage{mathrsfs}
\usepackage[all]{xy}
\usepackage{fancyhdr}
\usepackage{enumitem}
\usepackage{tabularx}
\usepackage{bibentry}
\usepackage{caption}
\usepackage{wrapfig}
\usepackage{csquotes}
\usepackage[compact]{titlesec}


\theoremstyle{plain}
\newtheorem{theorem}{Theorem}
\newtheorem{claim}[theorem]{Claim}
\newtheorem{lemma}[theorem]{Lemma}
\newtheorem{prop}[theorem]{Proposition}
\newtheorem{cor}[theorem]{Corollary}
\newtheorem{conj}[theorem]{Conjecture}

\theoremstyle{definition}
\newtheorem{question}{Question}
\newtheorem{questions}{Questions}
\newtheorem{definition}{Definition}
\newtheorem{pproblem}{Proposed Problem}

\theoremstyle{remark}
\newtheorem{remark}{Remark}
\newtheorem{notation}{Notation}
\titlespacing*{\subsection}
{0pt}{1.2mm}{1.5mm}



\DeclareMathOperator{\CP2}{\mathbb{C}P^2}
\DeclareMathOperator{\CPo}{\mathbb{C}P^1}
\DeclareMathOperator{\C}{\mathbb{C}}
\DeclareMathOperator{\R}{\mathbb{R}}
\DeclareMathOperator{\barCP2}{\overline{\mathbb{C}P^2}}
\DeclareMathOperator{\T}{\mathcal{T}}
\DeclareMathOperator{\He}{\mathcal{H}}
\DeclareMathOperator{\Lf}{\mathcal{L}}
\DeclareMathOperator{\St}{\mathcal{S}}
\DeclareMathOperator{\Z}{\mathbb{Z}}
\DeclareMathOperator{\Q}{\mathbb{Q}}
\DeclareMathOperator{\F}{\mathbb{F}}
\DeclareMathOperator{\bdry}{\partial}
\DeclareMathOperator{\last}{last}
\setlength{\parindent}{1.5em}
\setlength{\parskip}{.02mm}

\title{Research Statement}
\author{Miriam Kuzbary}

\pagestyle{fancy}\lhead{ \large{\bf Miriam Kuzbary}} \chead{ }
\rhead{{\large{\bf Teaching Statement}}} \lfoot{} \rfoot{\bf \thepage} \cfoot{}
\begin{document}
\thispagestyle{fancy}

When I teach, the most important goal I have is to help students develop their ability to analyze problems using logical reasoning and quantitative methods. These are physical skills which students cultivate through practice; however, this training requires motivation. To this end, my classroom techniques support students to feel personal ownership of the knowledge they gain in the course in order to develop their internal drive to practice mathematics. Moreover, in order for students to learn to use logical and quantitative skills effectively, they must become fluent in mathematical thinking and communication and thus I design my curriculum with this target in mind. Unfortunately, many students face structural, institutional, or personal barriers to connecting with mathematics in this meaningful way. The result is that some students come into a college math class unable to engage with the material and hence benefit very little from the course. This is particularly frustrating for students since nearly every college student must take at least one math class; therefore, I aim to provide a productive learning environment for my students regardless of their previous attitudes towards math by helping students engage more meaningfully with math.

I began training in mathematical pedagogy during my undergraduate career at the University of Texas at Dallas where I became certified to teach high school math through the inquiry-based UTeach program. This training has continued in graduate school at Rice University where I have taught five courses, two mini-courses in outreach programs, and assisted with classes ranging from calculus to graduate topology. Due to this mix of direct experience and professional development, I have designed my teaching methods to motivate students and increase their long-term benefit from taking a math class in the following ways. First, I \textbf{create meaningful and relevant mathematical experiences} throughout the course.  Moreover, I \textbf{foster collaborative problem solving} to help students engage even more deeply with the class, train in communicating their ideas, and practice the teamwork they will certainly experience in the workplace. Finally, I give students \textbf{multiple ways to demonstrate their understanding} while providing the scaffolding to meet high learning standards.


\subsection*{Making mathematics feel relevant.}Students are more willing to pay attention in class and practice new concepts if they feel a personal connection to either the material or the class itself. Students often have a fixed idea of what a successful math student looks and acts like and the student may not themselves fit this stereotype. This is particularly true for students from underrepresented backgrounds who may have never seen someone succeed in mathematics who they could see themselves in. Helping students feel ownership of the material and develop an identity as a person who can do math can help combat this inaccurate image. I start the semester with this perspective by surveying students about their majors as well as things they like and dislike about math. This survey starts the process of challenging student assumptions about their abilities by prompting them to analyze their own perspective before the course starts. It also lets me use their majors and interests to guide classroom examples. 

Throughout the semester I work to create a classroom environment where students feel the information is personally relevant to them.  For example, I structure each lecture so students feel as if they are discovering the concepts and techniques for themselves. In linear algebra, I guide the class in deriving the definitions of eigenvalues and eigenvectors while prompting them to interpret what is happening both algebraically and geometrically. As a result, on homework and exams my students not only perform computations accurately but demonstrate a deep understanding of why the computations they did were the correct thing to do. This student-centered approach models logical processes for students; not only preparing them for solving problems in a future job but introducing them to the living nature of mathematics to stimulate their interest in math research.

In order to support the development of my students into independent logical thinkers, I incorporate research skills into my classes with projects. In Pre-Calculus, my students learned trigonometric identities through a project where they analyzed the error between plugging values into simplified and unsimplified trigonometric expressions and presented the resulting data with a discussion on error propagation in computer programs. This hands-on approach increases student retention and encourages students to intellectually invest more in the course. As a result, students with a broad range of final grades in my class have communicated to me that my class was the first math class they had ever enjoyed. Though my courses emphasize learning math through a theoretical lens, I also hint at applications to ensure students feel the material is relevant to their existing interests. This provides a starting point for students to explore their own connections between what they are learning and how they might use specific techniques.  Consequently, one electrical engineering student emailed after the class was over, ``I recently read a paper for my research and they started going into the linear algebra. [Because of] your class, I felt like I really understood exactly what they were getting at.''


 \subsection*{Collaborative problem solving.} Students benefit from solving problems together; they are often more invested in course ideas and can better internalize concepts when they can explain them to their peers. This also prepares students for the team projects they will likely encounter in their future careers. Small group discussion also creates space for students from a variety of backgrounds to contribute to the class in meaningful and satisfying ways while preventing a small number of students from dominating the classroom.  Thus, I set up the course as a semester-long conversation about mathematics where the students and I test out ideas together.
 
In a typical lecture, I guide several different rounds of group work. Students work on basic examples such as performing a straightforward line integral on their own, discuss the results in pairs, and volunteer results. For deeper problems such as determining all possible subspaces of $\mathbb{R}^3$, students split off into small groups to work while I circulate the room and probe each group for their ideas. This way, I can give them immediate feedback and assess their progress holistically.  After talking things through together, each group designates a representative to report their discussion back to the whole class. In larger classes, I solicit group responses using clickers or colored notecards. Such techniques allow every student to contribute meaningfully and prevents small number of students from dominating the classroom. 
  
As students get used to this lecture style throughout the semester it is clear to see their mathematical maturity and confidence in their abilities improve. They become more comfortable engaging with mathematics in the supportive environment we create together in class since they are immediately responsible for talking about ideas and are able to work through concepts . One of my students in Calculus II stated in a course evaluation, ``She created a respectful, busy class atmosphere that was very easy to learn in.'' 


\subsection*{Scaffolding and multiple assessment techniques.} Because my lectures emphasize growth rather than immediate perfection, I am able to hold my students to high standards while giving them the tools to meet these standards. I provide different levels of scaffolding as course difficulty increases; in calculus II, I provide fill-in-the-blank lecture notes both to model organized note taking and to encourage students to critically analyze how they determine what information is important. In higher level classes like linear algebra, I do not give them notes and I let students lead class discussions more as the semester progresses. Regardless of the course, I pause throughout each lecture to give students mottos to help guide their synthesis of the lectures. Assessment in my courses includes online, immediately graded homework for practicing straightforward computations and more difficult written homework for testing the material on a deeper level which I encourage them to work together on and grade for content and clarity of exposition. I also ask them to submit write-ups from class discussions. 

\end{document}

%Additionally, I hold office hours throughout the week and meet with students by appointment; however, I also guide them to help each other during office hours and class discussions as I believe they learn more by explaining their thought process to their peers. Student feedback through evaluations has confirmed to me this combination of techniques is effective; moreover, on joint exams my students were more adept than other sections at answering deep, conceptual questions. When I design homework and exams, I construct questions to assess students' computation skills, understanding, and ability to extend their knowledge in new situations at various levels to give them accurate feedback on their progress in the course. However, I am careful to only test on the material they have learned in my course and not base their grades on what they may remember from other classes. I also guide students to ample review material when material from previous classes is necessary; this is particularly important for non-traditional students and for students who are motivated but may be less prepared for the course material than their peers. In every course syllabus I tell students that doing math is a physical skill they can improve at by practicing.  In this way, I challenge the assumptions of students who may believe they are simply not innately good at math and encourage a growth mindset. 

 %Meaningful group work models logical processes involved in applying mathematical thinking to new problems and supports the students' development as independent thinkers.
 %She was always enthusiastic in class, which made the class more enthusiastic as a whole.
%I design classroom discussion questions to be open-ended enough to encourage critical analysis instead of rote memorization of algorithms. For example, in linear algebra I ask students to work together to figure out what all possible subspaces of $3$-dimensional euclidean space can look like and in multivariable calculus I ask students if a line integral always depends on the path the integral is taken over. As I circulate the room, I prompt students to test out small examples and probe their underlying assumptions about the topic. As a result, in my most recent multivariable calculus class I even had students who came into the class insisting they were unable to do math bring up ideas to me in office hours demonstrating an extremely sophisticated understanding of the advanced calculus concepts in the class. It is important to me that each student finishes my course with the feeling that they have learned something valuable independent of the final grade on their transcript. Though these techniques are sometimes not employed in upper level math classes, I have experienced graduate classes with a group discussion component and believe this is an effective strategy for improving engagement, retention, and ownership of the material in all levels of math. , my students demonstrate longer retention of major concepts and a stronger sense of ownership of the material in their exams and in course evaluations.  Moreover, 
 
 
 
 
 
 
% I state in the syllabus, (PARAPHRASE QUOTE)
%
%\begin{displayquote}``Learning math is like playing a new sport; you need to practice regularly and do different types of drills in order to become competent. Our class time is your opportunity to make mistakes and ask questions. Do not be afraid of sounding silly! During our lectures we will have many conversations about what we are learning, so come to class expecting that you will be both contributing to the discussion and taking away something interesting to think about.'' \end{displayquote} by using an inquiry-based structure and hinting at applications while being mindful of the scope of the course

%For instance, I taught a small linear algebra course over the summer where the enrolled students were particularly advanced and all majoring in electrical engineering or computer science. Throughout the course I pointed out when certain topics would appear in classes in their major and solicited ideas from them about how the theoretical concepts in class might explain or help a certain situation they had encountered in other courses. At the end of the course, I chose to introduce singular value decompositions and their utility in applications based on the research interests of my students.

%It also allows me to use their majors and interests to occasionally incorporate relevant topics and examples into the curriculum.